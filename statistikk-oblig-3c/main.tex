\documentclass[12pt]{article}
\usepackage[utf8]{inputenc}
\usepackage[T1]{fontenc}
\usepackage[norsk]{babel} % For norsk språk
\usepackage{graphicx} % For bilder
\usepackage{lipsum} % Genererer fylltekst
\usepackage{amsmath}
\usepackage{hyperref}
\usepackage{float}

\title{Oblig 3c}
\author{Gormery K. Wanjiru}
\date{\today}

\begin{document}

\maketitle

\newpage
\tableofcontents

\newpage

\section{(15\%) kap. 17: oppgave 1.c}
\section{(15\%) kap. 17: oppgave 1.d}
\section{Terningdropp-oppgaven: (Totalt 50\%)}
\subsection{(5\%) Tegn et diagram med samtlige datapunkter, og legg på den lineære regresjonslinjen.}
\subsection{ $(15\%)$ Bruk nøytrale prior hyperparametre, og finn posterior og prediktive sannsynlighetsfordelinger, det vil si, sannsynlighetsfordelinger for $\tau$, $b$, $y(x)$ og $Y^+(x)$.}
\subsection{(5\%) Finn et 80\% kredibilitetsintervall (intervallestimat) for stigningstallet $b$.}
\subsection{(5\%) Finn et 80\% kredibilitetsintervall (intervallestimat) for standardavviket $\sigma$. (Hint: Bruk verdiene fra $\tau$ og regn om ved å bruke at $\tau = \frac{1}{\sigma^2}$)}
\subsection{(5\%) Finn et 80\% kredibilitetsintervall (intervallestimat) for $y(x)$.}
\subsection{(5\%) 80\% intervallestimatet for $y(x)$ er funksjoner av $x$, og en kurve over, og en under regresjonslinjen. Plott disse kurvene inn sammen med regresjonslinjen.}
\subsection{(5\%) Finn verdien $R^2 = \frac{SS_y - SS_e}{SS_y}$. Dette tallet forteller hvor stor del av variasjonen i $y$ som kan forklares av linja $y = a + bx$. For de av dere som bruker dataverktøy for å finne dette: angi hvordan dere fant det.}
\subsection{(5\%) Finn $R^2$ for regresjonen mellom $z$ (utfall på terningen) og $x$ (dropphøyde). Kommenter hva forskjellen mellom $R^2$ for $y$ og $R^2$ for $z$ sier oss.}
\section{(Totalt 20\%) Følgende R-kode vil plukke ut et utvalg av observasjonene.}
\subsection*{R kode}
\begin{verbatim}
  antall_rader = dim(dropp_df)[1]
  N = 20 # (Eksempel; se oppgavene)
  utvalg = sort(sample(1:antall_rader,N)) # Sortering er ikke nøvendig
  utvalg # men du får da se hvilke rader som er plukket ut
  ny_dropp_df = dropp_df[utvalg,] # Dette er kjernen; plukker ut radene
  rownames(ny_dropp_df)=1:N # Lurt hvis du skal kjøre for-løkke.
  ny_dropp_df # Ikke nødvendig, men du får se den nye data-rammen.
\end{verbatim}

\subsection{(5\%) Kjør 50 runder, og bruk $N = 15$. For hver runde, gjør oppgave 3a, men tegn regresjonslinjene sammen, i samme graf. Hva ser du?}
\subsection{(5\%) Kjør en runde med $N$ henholdsvis lik 5, 15, 50 og 200. For hver runde, gjør oppgavene 3c og 3d. Hva ser du?}
\subsection{(10\%) Kjør en runde med $N$ henholdsvis lik 5, 15, 50 og 200. For hver runde, gjør oppgaven 3f. Tegnes i hvert sitt diagram. Hva ser du?}

\newpage
\section*{Vedlegg}
\addcontentsline{toc}{section}{Vedlegg}
\subsection*{Vedlegg A}
\addcontentsline{toc}{subsection}{Vedlegg A}

\newpage
\begin{thebibliography}{9}

  \bibitem{referanse1}
  \url{https://tma4245.math.ntnu.no/viktige-diskrete-fordelinger/poissonprosess-og-poissonfordeling}
  \textit{NTNU}
\end{thebibliography}
\end{document}

