\documentclass{article}
\usepackage{graphicx}
\usepackage{float}
\usepackage{amsmath}

\begin{document}

\title{Analyse av Seigmenn Strekningen}
\author{Gormery K. Wanjiru}
\date{\today}
\maketitle

\section{Introduksjon}
Denne rapporten presenterer resultatene av strekningstester utført på seigmenn (laban). Dataene er analysert ved hjelp av statistiske metoder for å forstå de generelle egenskapene til seigmenns elastisitet.

\section{Metode}
Data ble samlet inn og bearbeidet i R. Følgende prosedyrer ble utført:
\begin{itemize}
    \item Innlesing av data fra en CSV-fil.
    \item Erstatning av manglende verdier med 0 for å unngå feil i beregningene.
    \item Beregning av kumulativ frekvens.
    \item Beregning av gjennomsnitt, median, typetall (modus) og standardavvik.
\end{itemize}

For å beregne både populasjonsstandardavviket og utvalgsstandardavviket, brukte vi følgende formler:
\begin{itemize}
    \item Populasjonsstandardavvik: 
    \[
    \sigma = \sqrt{\frac{1}{N}\sum_{i=1}^{N}(x_i - \mu)^2}
    \]
    \item Utvalgsstandardavvik: 
    \[
    s = \sqrt{\frac{1}{N-1}\sum_{i=1}^{N}(x_i - \bar{x})^2}
    \]
\end{itemize}

\section{Resultater}
Histogrammet og det kumulative frekvensdiagrammet ble generert. Statistiske mål middelverdi, median, typetall, og begge typer standardavvik (populasjonsstandardavvik og utvalgsstandardavvik) ble beregnet.

\subsection{Histogram}
\begin{figure}[H]
    \centering
    \includegraphics[width=0.8\textwidth]{Rplot01.png}
    \caption{Histogram med tydelig markerte statistiske mål}
\end{figure}

\subsection{Kumulativt Frekvensdiagram}
\begin{figure}[H]
    \centering
    \includegraphics[width=0.8\textwidth]{Rplot.png}
    \caption{Kumulativt Frekvensdiagram med tydelig markerte statistiske mål}
\end{figure}

\section{Diskusjon}
Gjennom analysen ble det funnet at seigmenn viser en bestemt tendens i strekningen med en gjennomsnittlig verdi på X, en median på Y, en modus på Z, et populasjonsstandardavvik på P, og et utvalgsstandardavvik på Q. Disse målene gir innsikt i hvordan seigmenn oppfører seg under strekk.

For å beregne disse målene brukte jeg følgende matematiske formler:
\begin{itemize}
    \item Middelverdi: $\bar{x} = \frac{1}{n}\sum_{i=1}^{n} x_i$ og fant at $\bar{x} = 11.42$.
    \item Median: Verdi i midten av datasettet når det er sortert. Fant at medianen er 11.5.
    \item Typetall: Den mest frekvente verdien i datasettet. Fant at typetallet er 11.
    \item Populasjonsstandardavvik: 
    \[
    \sigma = \sqrt{\frac{1}{N}\sum_{i=1}^{N}(x_i - \mu)^2}
    \] 
    og fant at $\sigma = 0.852$.
    \item Utvalgsstandardavvik: 
    \[
    s = \sqrt{\frac{1}{N-1}\sum_{i=1}^{N}(x_i - \bar{x})^2}
    \]
    og fant at $s = 0.862$.
\end{itemize}

\end{document}
