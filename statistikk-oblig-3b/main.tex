\documentclass[12pt]{article}
\usepackage[utf8]{inputenc}
\usepackage[T1]{fontenc}
\usepackage[norsk]{babel} % For norsk språk
\usepackage{graphicx} % For bilder
\usepackage{lipsum} % Genererer fylltekst
\usepackage{amsmath} % For matematiske formler
\usepackage{hyperref} % For hyperlenker
\usepackage{float}

\title{Oblig 3b}
\author{Gormery K. Wanjiru}
\date{\today}

\begin{document}

\maketitle

\newpage
\tableofcontents

\newpage
\section{Begreper}
\subsection{Hva er en Poisson-prosess?}
En poissonprosess er en prosess der hendelser av en bestemt type skjer ifølge spesifiserte sannsynlighetslover, blant annet at antall hendelser som skje i ikke-overlappende intervaller er uavhengige av hverandre.
dise egenskaper er:
\begin{enumerate}
  \item \( N(0) = 0 \). Initialtilstand: Antall hendelser ved starten er null
  \item For enhver \( 0 \leq t_1 < t_2 \leq t_3 < t_4 \) er \( N(t_2) - N(t_1) \) og \( N(t_4) - N(t_3) \) uavhengige stokastiske variabler. Uavhengighet: Antallet hendelser som skjer i ikke-overlappende intervaller er uavhengige av hverandre.

  \item For enhver \( t \geq 0 \) og \( \Delta t > 0 \) har vi at
        \[
          P(N(t + \Delta t) - N(t) = 1) = \lambda \Delta t + o(\Delta t),
        \]
        \[
          P(N(t + \Delta t) - N(t) \geq 2) = o(\Delta t),
        \]
        der hver \( o(\Delta t) \) angir en funksjon som oppfyller
        \[
          \lim_{\Delta t \to 0} \frac{o(\Delta t)}{\Delta t} = 0.
        \]
        Sannsynligheten for at en hendelse skjer i et lite tidsintervall \({\Delta t}\) er proporsjonal med lengden av intervallet, og sannsynligheten for at mer enn én hendelse skjer i dette intervallet er neglisjerbar.

\end{enumerate}


\subsection{Hva er prediktiv fordeling?}
En prediktiv fordeling beskriver sannsynligheten for fremtidige observasjoner gitt eksisterende data. for eksempel i bayesian statistikk representerer den distribusjonen av mulige utfall for en fremtidig observasjon basert på en nåværende postulert modell og de observerte dataene.
\subsection{Hva er sammenhengen mellom gamma-fordeling og gamma-gamma-fordeling,
og hva er forskjellen på de to?}

Gamma-fordelingen er en kontinuerlig sannsynlighetsfordeling som ofte brukes til å modellere ventetiden til den 
\({k}\) -te hendelsen i en Poisson-prosess. Den er karakterisert ved to parametere, formparameteren 
\({k}\) og raten 
\({\lambda}\) , og den har en bred anvendelse i queuing teori, finans, og forsikringsmatematikk.


Gamma-gamma-fordelingen refererer vanligvis til en sammensatt fordeling hvor formparameteren i en gamma-fordeling selv følger en gamma-fordeling, 
ofte brukt i bayesianske hierarkiske modeller. Denne fordelingen brukes for å modellere over-dispersjon i data hvor variansen er større enn middelverdien, 
som er typisk i visse typer forsikringskrav eller kjøpsfrekvensdata.

Sammenhengen mellom dem ligger i at gamma-gamma-fordelingen utvider gamma-fordelingen ved å tillate at dens parametere selv er tilfeldige variabler
 med sine egne fordelinger, noe som fører til en mer fleksibel modell. Forskjellen er i denne fleksibiliteten og i den ytterligere nivå av usikkerhet som innføres ved å ha parametere som selv er stokastiske.
\section{Kapittel 13: oppgave 13d}

Gitt en prior normalfordeling for \(\mu\) med kjent \(\sigma\), hvor hyperparameterne for prioren er \(\kappa_0 = 0\), \(\mu_0 = 0\), og \(\sigma_0 = 200\). Det er observert \(n = 12\) med gjennomsnitt \(\bar{x} = 35723.7\) og vi har at \(\sigma = 3125\).

Posteriorfordelingen for \(\mu\) etter å ha observert dataene er også en normalfordeling. Posteriorens parametre, \(\mu_n\) og \(\sigma_n^2\), kan beregnes som følger:
\[\mu_n = \frac{\kappa_0\mu_0 + n\bar{x}}{\kappa_0 + n}\]
\[\sigma_n^2 = \frac{1}{\frac{1}{\sigma_0^2} + \frac{n}{\sigma^2}}\]

Setter vi inn verdiene får vi:
\[\mu_n = \frac{0 \cdot 0 + 12 \cdot 35723.7}{0 + 12} = 35723.7\]
\[\sigma_n^2 = \frac{1}{\frac{1}{200^2} + \frac{12}{3125^2}}\]

Den oppdaterte verdien av \(\sigma_n^2\) blir den inverse av summen av de inverse variansene.

Sannsynlighetene for \(\mu\) og prediksjonene for fremtidig observasjon \(X_{+}\) kan beregnes ved å bruke den oppdaterte posteriore normalfordelingen for \(\mu\).

\[P(\mu < 3125) = 99.48\%\]
\[P(X_{+} < 3125) = 76.15\%\]

% \subsection*{R kode}
% \begin{verbatim}
%   kappa_0 <- 0
%   mu_0 <- 0
%   sigma_0 <- 200
%   n <- 12
%   x_bar <- 35723.7
%   sigma <- 3125
  
%   mu_n <- (kappa_0 * mu_0 + n * x_bar) / (kappa_0 + n)
%   sigma_n_squared <- 1 / (1 / sigma_0^2 + n / sigma^2)
  
%   # Beregning av sannsynlighetene
%   p_mu <- pnorm(3125, mean = mu_n, sd = sqrt(sigma_n_squared))
%   p_Xt <- pnorm(3125, mean = mu_n, sd = sqrt(sigma_n_squared + sigma^2))
  
%   p_mu
%   p_Xt
  
% \end{verbatim}


\section*{Oppgave 14(c)}

Gitt at stokastiske variabelen \( X \) er normalfordelt, \( X \sim \mathcal{N}(5,3.1^2) \), og en nyttefunksjon \( u(x) \) som er definert som:

\[ u(x) = \begin{cases} 
-1 & \text{for } x < 3 \\
1.5 & \text{for } 3 \leq x < 6 \\
4 & \text{for } x \geq 6 
\end{cases} \]

Forventet nytte \( U \) kan beregnes ved å ta forventningen av \( u(X) \) med hensyn på fordelingen til \( X \):

\[ U = E[u(X)] = -1 \cdot P(X < 3) + 1.5 \cdot P(3 \leq X < 6) + 4 \cdot P(X \geq 6) \]

Ved å sette inn de relevante sannsynlighetene fra den normalfordelte stokastiske variabelen får vi:

\[ U = -1 \cdot \Phi\left(\frac{3-5}{3.1}\right) + 1.5 \cdot \left(\Phi\left(\frac{6-5}{3.1}\right) - \Phi\left(\frac{3-5}{3.1}\right)\right) + 4 \cdot \left(1 - \Phi\left(\frac{6-5}{3.1}\right)\right) \]

\[ U = -1 \cdot \Phi(-0.64516) + 1.5 \cdot (\Phi(0.32258) - \Phi(-0.64516)) + 4 \cdot (1 - \Phi(0.32258)) \]

\[ U \approx 2.1081 \]

% \subsection*{R kode}
% \begin{verbatim}
%   mu <- 5
%   sigma <- 3.1
  
%   # Beregn sannsynlighetene
%   p_X_less_3 <- pnorm(3, mean = mu, sd = sigma, lower.tail = TRUE)
%   p_X_between_3_6 <- pnorm(6, mean = mu, sd = sigma, lower.tail = TRUE) - p_X_less_3
%   p_X_more_6 <- 1 - pnorm(6, mean = mu, sd = sigma, lower.tail = TRUE)
  
%   # Beregn forventet nytte
%   U <- -1 * p_X_less_3 + 1.5 * p_X_between_3_6 + 4 * p_X_more_6
%   U
% \end{verbatim}


\section{De tre eksperimentene:}
\subsection{(a) Bernoulli}

\subsubsection{i. Intervallestimater}
For å regne ut 90\% kredibilitetsintervall for parameteren $\pi$ for alle 4 $n$-verdier, bruker vi Betafordelingen $\text{Beta}(a,b)$, hvor $a$ er antall suksesser pluss 1, og $b$ er antall feil pluss 1. For $n = 10, 100, 1000, 10000$, anta vi har observerte suksesser gitt i oppgaven.

Kredibilitetsintervallet for en Betafordeling er gitt ved:
\[
\text{CI} = \left[\text{Beta}^{-1}\left(\frac{\alpha}{2}; a, b\right), \text{Beta}^{-1}\left(1-\frac{\alpha}{2}; a, b\right)\right]
\]

hvor $\alpha = 0.1$ for et 90\% intervall.

\subsubsection{ii. Punktestimater}
Vi antar at for $n=100$ og $n=10000$, antallet suksesser (a) og feil (b) er gitt. Vi beregner punktestimatene for disse to n-verdiene.

For å gjøre dette, bruker vi R. R-kode for beregning av måleparameter (forventningsverdi) og varians for Betafordelingen for $n=100$ og $n=10000$.

\subsection{R Kode for Punktestimater}
\begin{verbatim}
# For n=100
a100 <- 70 # Antall suksesser + 1
b100 <- 31 # Antall feil + 1
mu100 <- a100 / (a100 + b100)
var100 <- (a100 * b100) / ((a100 + b100)^2 * (a100 + b100 + 1))

# For n=10000
a10000 <- 7000 # Antall suksesser + 1
b10000 <- 3001 # Antall feil + 1
mu10000 <- a10000 / (a10000 + b10000)
var10000 <- (a10000 * b10000) / ((a10000 + b10000)^2 * (a10000 + b10000 + 1))

# resultatene
print(c(mu100, var100))
print(c(mu10000, var10000))
\end{verbatim}

\subsubsection{iii. Hypotesetesting}
For å teste hypotesen $H_1: \pi > 0.5$ med signifikansnivå $\alpha = 0.1$, beregner vi $P(H_0)$ for alle de fire posterior-fordelingene. Dette kan gjøres ved å bruke den kumulative fordelingsfunksjonen for Betafordelingen i R.

\subsection{R Kode for Hypotesetesting}
\begin{verbatim}
#for å beregne P(H0) for de fire posterior-fordelingene
library(stats)

# Anta at a og b er definert som tidligere for hvert n
# Beregning for n=10
a10 <- 7; b10 <- 4
PH0_n10 <- 1 - pbeta(0.5, a10, b10)

# Beregning for n=100, a100 og b100 definert tidligere
PH0_n100 <- 1 - pbeta(0.5, a100, b100)

# Beregning for n=1000
a1000 <- 700; b1000 <- 301
PH0_n1000 <- 1 - pbeta(0.5, a1000, b1000)

# Beregning for n=10000, a10000 og b10000 definert tidligere
PH0_n10000 <- 1 - pbeta(0.5, a10000, b10000)

# resultatene
print(c(PH0_n10, PH0_n100, PH0_n1000, PH0_n10000))
\end{verbatim}


\newpage
\section*{Vedlegg}
\addcontentsline{toc}{section}{Vedlegg}
\subsection*{Vedlegg A}
\addcontentsline{toc}{subsection}{Vedlegg A}

\newpage
\begin{thebibliography}{9}

  \bibitem{referanse1}
  \url{https://tma4245.math.ntnu.no/viktige-diskrete-fordelinger/poissonprosess-og-poissonfordeling}
  \textit{NTNU}
\end{thebibliography}

\end{document}
