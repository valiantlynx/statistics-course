\documentclass{article}
\usepackage{enumitem}

\begin{document}

\title{Forprosjekt: Kulturelle og Geografiske Påvirkninger på Matpreferanser}
\author{Gormery k. Wanjiru}
\date{\today}

\maketitle

\section*{Prosjektidee}

Undersøke sammenhengen mellom matpreferanser, spesielt fett og fermentert mat, og kulturelle eller geografiske bakgrunner.

\section*{Problemstilling}

Hvordan påvirker kulturelle og geografiske faktorer individuelle preferanser for fett og fermentert mat?

\section*{Metode}

\begin{enumerate}[label=\arabic*.]
  \item \textbf{Datainnsamling:}
  \begin{itemize}
    \item Lag en spørreundersøkelse om kulturell bakgrunn, geografisk plassering og matpreferanser.
    \item Samle data fra et mangfoldig utvalg av respondenter. inkludert elever, lærer og eventuelt nettet.
  \end{itemize}
  
  \item \textbf{Statistisk Analyse:}
  \begin{itemize}
    \item Analyser dataene ved hjelp av statistiske metoder.
    \item Sammenlign resultater basert på kulturelle og geografiske kategorier.
  \end{itemize}
  
  \item \textbf{Resultater og Konklusjoner:}
  \begin{itemize}
    \item Presentér funnene og diskuter observasjoner og sammenhenger.
    \item Vurder hvordan kulturelle og geografiske påvirkninger kan ha formet matpreferansene.
  \end{itemize}
  
  \item \textbf{Diskusjon om Kulturelle Kontekster:}
  \begin{itemize}
    \item Ta opp tidligere forskning eller teorier om påvirkningen av kulturelle kontekster på matpreferanser.
    \item Reflekter over hvorfor visse matvarer er mer eller mindre populære i bestemte regioner eller kulturer.
  \end{itemize}
  
  \item \textbf{Refleksjon over Metode og Begrensninger:}
  \begin{itemize}
    \item Diskuter styrker og svakheter ved undersøkelsesmetoden.
    \item Identifiser eventuelle begrensninger i prosjektet og hvordan de kan ha påvirket resultatene.
  \end{itemize}
\end{enumerate}

\end{document}
